% Options for packages loaded elsewhere
\PassOptionsToPackage{unicode}{hyperref}
\PassOptionsToPackage{hyphens}{url}
\PassOptionsToPackage{dvipsnames,svgnames,x11names}{xcolor}
%
\documentclass[
  letterpaper,
  DIV=11,
  numbers=noendperiod]{scrreprt}

\usepackage{amsmath,amssymb}
\usepackage{iftex}
\ifPDFTeX
  \usepackage[T1]{fontenc}
  \usepackage[utf8]{inputenc}
  \usepackage{textcomp} % provide euro and other symbols
\else % if luatex or xetex
  \usepackage{unicode-math}
  \defaultfontfeatures{Scale=MatchLowercase}
  \defaultfontfeatures[\rmfamily]{Ligatures=TeX,Scale=1}
\fi
\usepackage{lmodern}
\ifPDFTeX\else  
    % xetex/luatex font selection
\fi
% Use upquote if available, for straight quotes in verbatim environments
\IfFileExists{upquote.sty}{\usepackage{upquote}}{}
\IfFileExists{microtype.sty}{% use microtype if available
  \usepackage[]{microtype}
  \UseMicrotypeSet[protrusion]{basicmath} % disable protrusion for tt fonts
}{}
\makeatletter
\@ifundefined{KOMAClassName}{% if non-KOMA class
  \IfFileExists{parskip.sty}{%
    \usepackage{parskip}
  }{% else
    \setlength{\parindent}{0pt}
    \setlength{\parskip}{6pt plus 2pt minus 1pt}}
}{% if KOMA class
  \KOMAoptions{parskip=half}}
\makeatother
\usepackage{xcolor}
\setlength{\emergencystretch}{3em} % prevent overfull lines
\setcounter{secnumdepth}{5}
% Make \paragraph and \subparagraph free-standing
\ifx\paragraph\undefined\else
  \let\oldparagraph\paragraph
  \renewcommand{\paragraph}[1]{\oldparagraph{#1}\mbox{}}
\fi
\ifx\subparagraph\undefined\else
  \let\oldsubparagraph\subparagraph
  \renewcommand{\subparagraph}[1]{\oldsubparagraph{#1}\mbox{}}
\fi

\usepackage{color}
\usepackage{fancyvrb}
\newcommand{\VerbBar}{|}
\newcommand{\VERB}{\Verb[commandchars=\\\{\}]}
\DefineVerbatimEnvironment{Highlighting}{Verbatim}{commandchars=\\\{\}}
% Add ',fontsize=\small' for more characters per line
\usepackage{framed}
\definecolor{shadecolor}{RGB}{241,243,245}
\newenvironment{Shaded}{\begin{snugshade}}{\end{snugshade}}
\newcommand{\AlertTok}[1]{\textcolor[rgb]{0.68,0.00,0.00}{#1}}
\newcommand{\AnnotationTok}[1]{\textcolor[rgb]{0.37,0.37,0.37}{#1}}
\newcommand{\AttributeTok}[1]{\textcolor[rgb]{0.40,0.45,0.13}{#1}}
\newcommand{\BaseNTok}[1]{\textcolor[rgb]{0.68,0.00,0.00}{#1}}
\newcommand{\BuiltInTok}[1]{\textcolor[rgb]{0.00,0.23,0.31}{#1}}
\newcommand{\CharTok}[1]{\textcolor[rgb]{0.13,0.47,0.30}{#1}}
\newcommand{\CommentTok}[1]{\textcolor[rgb]{0.37,0.37,0.37}{#1}}
\newcommand{\CommentVarTok}[1]{\textcolor[rgb]{0.37,0.37,0.37}{\textit{#1}}}
\newcommand{\ConstantTok}[1]{\textcolor[rgb]{0.56,0.35,0.01}{#1}}
\newcommand{\ControlFlowTok}[1]{\textcolor[rgb]{0.00,0.23,0.31}{#1}}
\newcommand{\DataTypeTok}[1]{\textcolor[rgb]{0.68,0.00,0.00}{#1}}
\newcommand{\DecValTok}[1]{\textcolor[rgb]{0.68,0.00,0.00}{#1}}
\newcommand{\DocumentationTok}[1]{\textcolor[rgb]{0.37,0.37,0.37}{\textit{#1}}}
\newcommand{\ErrorTok}[1]{\textcolor[rgb]{0.68,0.00,0.00}{#1}}
\newcommand{\ExtensionTok}[1]{\textcolor[rgb]{0.00,0.23,0.31}{#1}}
\newcommand{\FloatTok}[1]{\textcolor[rgb]{0.68,0.00,0.00}{#1}}
\newcommand{\FunctionTok}[1]{\textcolor[rgb]{0.28,0.35,0.67}{#1}}
\newcommand{\ImportTok}[1]{\textcolor[rgb]{0.00,0.46,0.62}{#1}}
\newcommand{\InformationTok}[1]{\textcolor[rgb]{0.37,0.37,0.37}{#1}}
\newcommand{\KeywordTok}[1]{\textcolor[rgb]{0.00,0.23,0.31}{#1}}
\newcommand{\NormalTok}[1]{\textcolor[rgb]{0.00,0.23,0.31}{#1}}
\newcommand{\OperatorTok}[1]{\textcolor[rgb]{0.37,0.37,0.37}{#1}}
\newcommand{\OtherTok}[1]{\textcolor[rgb]{0.00,0.23,0.31}{#1}}
\newcommand{\PreprocessorTok}[1]{\textcolor[rgb]{0.68,0.00,0.00}{#1}}
\newcommand{\RegionMarkerTok}[1]{\textcolor[rgb]{0.00,0.23,0.31}{#1}}
\newcommand{\SpecialCharTok}[1]{\textcolor[rgb]{0.37,0.37,0.37}{#1}}
\newcommand{\SpecialStringTok}[1]{\textcolor[rgb]{0.13,0.47,0.30}{#1}}
\newcommand{\StringTok}[1]{\textcolor[rgb]{0.13,0.47,0.30}{#1}}
\newcommand{\VariableTok}[1]{\textcolor[rgb]{0.07,0.07,0.07}{#1}}
\newcommand{\VerbatimStringTok}[1]{\textcolor[rgb]{0.13,0.47,0.30}{#1}}
\newcommand{\WarningTok}[1]{\textcolor[rgb]{0.37,0.37,0.37}{\textit{#1}}}

\providecommand{\tightlist}{%
  \setlength{\itemsep}{0pt}\setlength{\parskip}{0pt}}\usepackage{longtable,booktabs,array}
\usepackage{calc} % for calculating minipage widths
% Correct order of tables after \paragraph or \subparagraph
\usepackage{etoolbox}
\makeatletter
\patchcmd\longtable{\par}{\if@noskipsec\mbox{}\fi\par}{}{}
\makeatother
% Allow footnotes in longtable head/foot
\IfFileExists{footnotehyper.sty}{\usepackage{footnotehyper}}{\usepackage{footnote}}
\makesavenoteenv{longtable}
\usepackage{graphicx}
\makeatletter
\def\maxwidth{\ifdim\Gin@nat@width>\linewidth\linewidth\else\Gin@nat@width\fi}
\def\maxheight{\ifdim\Gin@nat@height>\textheight\textheight\else\Gin@nat@height\fi}
\makeatother
% Scale images if necessary, so that they will not overflow the page
% margins by default, and it is still possible to overwrite the defaults
% using explicit options in \includegraphics[width, height, ...]{}
\setkeys{Gin}{width=\maxwidth,height=\maxheight,keepaspectratio}
% Set default figure placement to htbp
\makeatletter
\def\fps@figure{htbp}
\makeatother
\newlength{\cslhangindent}
\setlength{\cslhangindent}{1.5em}
\newlength{\csllabelwidth}
\setlength{\csllabelwidth}{3em}
\newlength{\cslentryspacingunit} % times entry-spacing
\setlength{\cslentryspacingunit}{\parskip}
\newenvironment{CSLReferences}[2] % #1 hanging-ident, #2 entry spacing
 {% don't indent paragraphs
  \setlength{\parindent}{0pt}
  % turn on hanging indent if param 1 is 1
  \ifodd #1
  \let\oldpar\par
  \def\par{\hangindent=\cslhangindent\oldpar}
  \fi
  % set entry spacing
  \setlength{\parskip}{#2\cslentryspacingunit}
 }%
 {}
\usepackage{calc}
\newcommand{\CSLBlock}[1]{#1\hfill\break}
\newcommand{\CSLLeftMargin}[1]{\parbox[t]{\csllabelwidth}{#1}}
\newcommand{\CSLRightInline}[1]{\parbox[t]{\linewidth - \csllabelwidth}{#1}\break}
\newcommand{\CSLIndent}[1]{\hspace{\cslhangindent}#1}

\KOMAoption{captions}{tableheading}
\makeatletter
\@ifpackageloaded{tcolorbox}{}{\usepackage[skins,breakable]{tcolorbox}}
\@ifpackageloaded{fontawesome5}{}{\usepackage{fontawesome5}}
\definecolor{quarto-callout-color}{HTML}{909090}
\definecolor{quarto-callout-note-color}{HTML}{0758E5}
\definecolor{quarto-callout-important-color}{HTML}{CC1914}
\definecolor{quarto-callout-warning-color}{HTML}{EB9113}
\definecolor{quarto-callout-tip-color}{HTML}{00A047}
\definecolor{quarto-callout-caution-color}{HTML}{FC5300}
\definecolor{quarto-callout-color-frame}{HTML}{acacac}
\definecolor{quarto-callout-note-color-frame}{HTML}{4582ec}
\definecolor{quarto-callout-important-color-frame}{HTML}{d9534f}
\definecolor{quarto-callout-warning-color-frame}{HTML}{f0ad4e}
\definecolor{quarto-callout-tip-color-frame}{HTML}{02b875}
\definecolor{quarto-callout-caution-color-frame}{HTML}{fd7e14}
\makeatother
\makeatletter
\makeatother
\makeatletter
\@ifpackageloaded{bookmark}{}{\usepackage{bookmark}}
\makeatother
\makeatletter
\@ifpackageloaded{caption}{}{\usepackage{caption}}
\AtBeginDocument{%
\ifdefined\contentsname
  \renewcommand*\contentsname{Table of contents}
\else
  \newcommand\contentsname{Table of contents}
\fi
\ifdefined\listfigurename
  \renewcommand*\listfigurename{List of Figures}
\else
  \newcommand\listfigurename{List of Figures}
\fi
\ifdefined\listtablename
  \renewcommand*\listtablename{List of Tables}
\else
  \newcommand\listtablename{List of Tables}
\fi
\ifdefined\figurename
  \renewcommand*\figurename{Figure}
\else
  \newcommand\figurename{Figure}
\fi
\ifdefined\tablename
  \renewcommand*\tablename{Table}
\else
  \newcommand\tablename{Table}
\fi
}
\@ifpackageloaded{float}{}{\usepackage{float}}
\floatstyle{ruled}
\@ifundefined{c@chapter}{\newfloat{codelisting}{h}{lop}}{\newfloat{codelisting}{h}{lop}[chapter]}
\floatname{codelisting}{Listing}
\newcommand*\listoflistings{\listof{codelisting}{List of Listings}}
\makeatother
\makeatletter
\@ifpackageloaded{caption}{}{\usepackage{caption}}
\@ifpackageloaded{subcaption}{}{\usepackage{subcaption}}
\makeatother
\makeatletter
\@ifpackageloaded{tcolorbox}{}{\usepackage[skins,breakable]{tcolorbox}}
\makeatother
\makeatletter
\@ifundefined{shadecolor}{\definecolor{shadecolor}{rgb}{.97, .97, .97}}
\makeatother
\makeatletter
\makeatother
\makeatletter
\makeatother
\ifLuaTeX
  \usepackage{selnolig}  % disable illegal ligatures
\fi
\IfFileExists{bookmark.sty}{\usepackage{bookmark}}{\usepackage{hyperref}}
\IfFileExists{xurl.sty}{\usepackage{xurl}}{} % add URL line breaks if available
\urlstyle{same} % disable monospaced font for URLs
\hypersetup{
  pdftitle={Introduction to R for Clinical Psychology},
  pdfauthor={Christopher J Wilson},
  colorlinks=true,
  linkcolor={blue},
  filecolor={Maroon},
  citecolor={Blue},
  urlcolor={Blue},
  pdfcreator={LaTeX via pandoc}}

\title{Introduction to R for Clinical Psychology}
\author{Christopher J Wilson}
\date{2025-01-06}

\begin{document}
\maketitle
\ifdefined\Shaded\renewenvironment{Shaded}{\begin{tcolorbox}[enhanced, breakable, frame hidden, boxrule=0pt, interior hidden, borderline west={3pt}{0pt}{shadecolor}, sharp corners]}{\end{tcolorbox}}\fi

\renewcommand*\contentsname{Table of contents}
{
\hypersetup{linkcolor=}
\setcounter{tocdepth}{2}
\tableofcontents
}
\bookmarksetup{startatroot}

\hypertarget{welcome}{%
\chapter*{Welcome}\label{welcome}}
\addcontentsline{toc}{chapter}{Welcome}

\markboth{Welcome}{Welcome}

The purpose of this site/book is to introduce you to R and R Studio for
use in Clinical Psychology Research. Before you begin, there are some
important things to know:

\begin{enumerate}
\def\labelenumi{\arabic{enumi}.}
\item
  This site is not a comprehensive guide to everything that can be done
  in R. There are many resources available for learning R, and this site
  is just a starting point. It will cover the basics of R and R Studio,
  and will provide examples of how to use R for common tasks in clinical
  psychology research.
\item
  This site is not a comprehensive guide to statistics. It will cover
  some key statistical concepts and how to perform them in R, but it is
  not a compete substitute for a statistics textbook or course. The goal
  of this content is to cover some key statistical concepts that are
  pertinent to clinical psychology research. It is expected that you
  have some background in statistics, and that you are familiar with
  basic concepts such as hypothesis testing, p-values, and confidence
  intervals.
\item
  It is important to note that R is a powerful tool, but it can be
  complex and challenging to learn. It is normal to feel overwhelmed at
  times, and it is important to be patient with yourself as you learn.
  The more you practice and use R, the more comfortable you will become
  with it. Set your expectations accordingly, and remember that learning
  R is a process that takes a long time and should continue throughout
  your career. This is just your starting point.
\item
  This site is a work in progress. I am constantly updating, adding new
  content and removing some less useful elements, so please check back
  regularly for updates. If you have any feedback or suggestions for
  content, please let me know.
\end{enumerate}

\hypertarget{tidyverse}{%
\section*{The tidyverse}\label{tidyverse}}
\addcontentsline{toc}{section}{The tidyverse}

\markright{The tidyverse}

\begin{tcolorbox}[enhanced jigsaw, opacitybacktitle=0.6, rightrule=.15mm, coltitle=black, bottomrule=.15mm, breakable, titlerule=0mm, leftrule=.75mm, toprule=.15mm, arc=.35mm, colback=white, colframe=quarto-callout-important-color-frame, opacityback=0, bottomtitle=1mm, toptitle=1mm, title=\textcolor{quarto-callout-important-color}{\faExclamation}\hspace{0.5em}{This site/book uses the tidyverse set of packages}, left=2mm, colbacktitle=quarto-callout-important-color!10!white]

The tidyverse is a collection of R packages designed for to make data
manipulation and visualization. easier in R. The tidyverse is a powerful
set of tools for data analysis, and it is widely used in the R
community. It is assumed that you will have the tidyverse installed and
loaded for the examples in this site/book. If you do not have the
tidyverse installed, you can install it by running the following code in
the R console:

\begin{Shaded}
\begin{Highlighting}[]

\FunctionTok{install.packages}\NormalTok{(}\StringTok{"tidyverse"}\NormalTok{)}
\end{Highlighting}
\end{Shaded}

You only need to install the package on to your machine once. Once you
have installed the tidyverse, you can load it by running the following
code in the R console:

\begin{Shaded}
\begin{Highlighting}[]

\FunctionTok{library}\NormalTok{(tidyverse)}
\end{Highlighting}
\end{Shaded}

The \texttt{library()} function is used to load packages in R. You will
need to load the tidyverse package at the beginning of each R session in
which you want to use it.

\end{tcolorbox}

\bookmarksetup{startatroot}

\hypertarget{introduction-to-r-and-r-studio}{%
\chapter{Introduction to R and R
Studio}\label{introduction-to-r-and-r-studio}}

\hypertarget{what-are-r-and-r-studio}{%
\section{What are R and R Studio?}\label{what-are-r-and-r-studio}}

\begin{tcolorbox}[enhanced jigsaw, opacitybacktitle=0.6, rightrule=.15mm, coltitle=black, bottomrule=.15mm, breakable, titlerule=0mm, leftrule=.75mm, toprule=.15mm, arc=.35mm, colback=white, colframe=quarto-callout-tip-color-frame, opacityback=0, bottomtitle=1mm, toptitle=1mm, title=\textcolor{quarto-callout-tip-color}{\faLightbulb}\hspace{0.5em}{At the end of this section, you will be able to:}, left=2mm, colbacktitle=quarto-callout-tip-color!10!white]

\begin{itemize}
\tightlist
\item
  Download and install R and R Studio
\item
  Understand the basic layout of R Studio
\item
  Describe some of the differences between SPSS and R
\end{itemize}

\end{tcolorbox}

\hypertarget{downloading-and-installing-r-and-r-studio}{%
\subsection{Downloading and installing R and R
Studio}\label{downloading-and-installing-r-and-r-studio}}

To get started with R Studio, you need to download and install two
pieces of software:

\begin{enumerate}
\def\labelenumi{\arabic{enumi}.}
\tightlist
\item
  \textbf{R}: The base software that you will use to write and run code.
\item
  \textbf{R Studio}: An integrated development environment (IDE) that
  makes it easier to write and run code in R.
\end{enumerate}

Click on these links to download:

\begin{itemize}
\tightlist
\item
  \href{https://cran.r-project.org/}{R project}
\item
  \href{https://rstudio.com/}{RStudio}
\end{itemize}

\hypertarget{the-r-studio-layout}{%
\subsection{The R Studio layout}\label{the-r-studio-layout}}

When you open R Studio, you will see a screen that looks like this:

\begin{figure}

{\centering \includegraphics{images/rstudio_ide.png}

}

\caption{R Studio IDE}

\end{figure}

Briefly, the different panes in R Studio are:

\begin{itemize}
\tightlist
\item
  \textbf{Console}: You can write and run code in this pane. However, it
  is best practice to write code in a script. You will see output from
  your code in the console.
\item
  \textbf{Environment/History}: This pane shows you the objects that you
  have created in R, and the history of the commands that you have run.
\item
  \textbf{Files/Plots/Packages/Help}: These panes allow you to navigate
  your files, view plots, manage packages, and access help
  documentation.
\end{itemize}

You will learn more about these panes as you work through the course.

\begin{figure}

{\centering \includegraphics{images/rstudio1.png}

}

\caption{R Studio IDE}

\end{figure}

\hypertarget{differences-between-spss-and-r}{%
\subsection{Differences between SPSS and
R}\label{differences-between-spss-and-r}}

R is a statistical programming language, while SPSS is a point-and-click
software package. This means that in R, you write code to perform tasks,
while in SPSS, you click buttons and select options from menus.

This can take some getting used to, but there are many advantages to
using R:

\begin{itemize}
\tightlist
\item
  \textbf{Reproducibility}: You can save your code and rerun it at any
  time, ensuring that your analysis is reproducible.
\item
  \textbf{Flexibility}: You can write code to perform any task you like,
  rather than being limited to the options available in a menu.
\item
  \textbf{Community}: R has a large and active community of users who
  share code and help each other to solve problems.
\end{itemize}

With R, you won't manipulate your source data files. Instead, you load
the data into R and manipulate it in R. This means that you can always
go back to your original data and start again if you need to.

\hypertarget{no-more-point-and-click---the-r-workflow.}{%
\section{No more ``point and click''! - the R
workflow.}\label{no-more-point-and-click---the-r-workflow.}}

\begin{tcolorbox}[enhanced jigsaw, opacitybacktitle=0.6, rightrule=.15mm, coltitle=black, bottomrule=.15mm, breakable, titlerule=0mm, leftrule=.75mm, toprule=.15mm, arc=.35mm, colback=white, colframe=quarto-callout-tip-color-frame, opacityback=0, bottomtitle=1mm, toptitle=1mm, title=\textcolor{quarto-callout-tip-color}{\faLightbulb}\hspace{0.5em}{At the end of this section, you will be able to:}, left=2mm, colbacktitle=quarto-callout-tip-color!10!white]

\begin{itemize}
\tightlist
\item
  Open a new script in R Studio
\item
  Write and run code in a script
\item
  Save a script for later use
\end{itemize}

\end{tcolorbox}

\hypertarget{using-scripts-in-r-studio}{%
\subsection{Using scripts in R Studio}\label{using-scripts-in-r-studio}}

When you work in R, you will write code in a script. This is a text file
that contains the code that you want to run. You can write and run code
in the console, but it is best practice to write code in a script. This
allows you to save your code and run it again later IT also makes it
easier to see what you have done.

To open a new script in R Studio, click on
\texttt{File\ \textgreater{}\ New\ File\ \textgreater{}\ R\ Script}.
This will open a new script in the top-left pane of R Studio.

You can write code in the script, and then run it by selecting the code
that you want to run and clicking the \texttt{Run} button at the top of
the script pane. You can also run code by pressing
\texttt{Ctrl\ +\ Enter} on your keyboard.

To save your script, click on \texttt{File\ \textgreater{}\ Save\ As...}
and save the file with a \texttt{.R} extension.

\begin{tcolorbox}[enhanced jigsaw, opacitybacktitle=0.6, rightrule=.15mm, coltitle=black, bottomrule=.15mm, breakable, titlerule=0mm, leftrule=.75mm, toprule=.15mm, arc=.35mm, colback=white, colframe=quarto-callout-important-color-frame, opacityback=0, bottomtitle=1mm, toptitle=1mm, title=\textcolor{quarto-callout-important-color}{\faExclamation}\hspace{0.5em}{Organising your work}, left=2mm, colbacktitle=quarto-callout-important-color!10!white]

It is good practice to keep your work organised by putting your scripts,
data, and other files in a folder on your computer, for each project
that you work on.

RStudio also allows the creation of projects. You can create a new
project in R Studio by clicking on
\texttt{File\ \textgreater{}\ New\ Project...}. This will create a new
folder on your computer where you can save your scripts, data, and other
files. If you save your script in the project folder, you can easily
access it by opening the project in R Studio. If you use projects, be
aware that R Studio will load the last project you worked on when you
open the software.

\end{tcolorbox}

\hypertarget{objects-functions-and-packages-in-r}{%
\section{Objects, functions and packages in
R}\label{objects-functions-and-packages-in-r}}

\begin{tcolorbox}[enhanced jigsaw, opacitybacktitle=0.6, rightrule=.15mm, coltitle=black, bottomrule=.15mm, breakable, titlerule=0mm, leftrule=.75mm, toprule=.15mm, arc=.35mm, colback=white, colframe=quarto-callout-tip-color-frame, opacityback=0, bottomtitle=1mm, toptitle=1mm, title=\textcolor{quarto-callout-tip-color}{\faLightbulb}\hspace{0.5em}{At the end of this section, you will be able to:}, left=2mm, colbacktitle=quarto-callout-tip-color!10!white]

\begin{itemize}
\tightlist
\item
  Create objects in R
\item
  Use functions in R to perform tasks
\item
  Install and load packages in R
\end{itemize}

\end{tcolorbox}

\hypertarget{what-are-objects}{%
\subsection{What are objects?}\label{what-are-objects}}

In R, you can create objects to store data. For example, you can create
an object called \texttt{numbers} that contains a set of numbers like
this:

\begin{Shaded}
\begin{Highlighting}[]
\NormalTok{numbers }\OtherTok{\textless{}{-}} \FunctionTok{c}\NormalTok{(}\DecValTok{1}\NormalTok{, }\DecValTok{2}\NormalTok{, }\DecValTok{3}\NormalTok{, }\DecValTok{4}\NormalTok{, }\DecValTok{5}\NormalTok{)}
\end{Highlighting}
\end{Shaded}

Breaking this code down:

\begin{itemize}
\item
  \texttt{numbers} is the name of the object that you are creating.
\item
  \texttt{\textless{}-} is the assignment operator. It assigns the value
  on the right-hand side of the operator to the object on the left-hand
  side.
\item
  \texttt{c(1,\ 2,\ 3,\ 4,\ 5)} is the data that you are assigning to
  the object. In this case, it is a set of numbers.
\end{itemize}

When you run this code, R will create an object called \texttt{numbers}
that contains the numbers 1, 2, 3, 4, and 5. You will be ab` le to see
the object in the Environment pane in R Studio.

You can then use the object in your code, instead of typing out the data
each time (see Section~\ref{sec-functions} for example).

\hypertarget{sec-functions}{%
\subsection{What are functions?}\label{sec-functions}}

Functions are code that have been written to perform a specific task.
You can use functions in R to perform tasks like reading data into R,
summarising data, and creating plots.

For example, the \texttt{mean()} function calculates the mean of a set
of numbers. You can use the \texttt{mean()} function like this:

\begin{Shaded}
\begin{Highlighting}[]
\NormalTok{numbers }\OtherTok{\textless{}{-}} \FunctionTok{c}\NormalTok{(}\DecValTok{1}\NormalTok{, }\DecValTok{2}\NormalTok{, }\DecValTok{3}\NormalTok{, }\DecValTok{4}\NormalTok{, }\DecValTok{5}\NormalTok{)}

\FunctionTok{mean}\NormalTok{(numbers)}
\end{Highlighting}
\end{Shaded}

Functions in R have a name, followed by parentheses. You can pass
arguments to the function inside the parentheses. In this case, the
\texttt{mean()} function takes a set of numbers as an argument, and
returns the mean of those numbers.

To learn more about a function, you can use the \texttt{help()}
function. For example, to learn more about the \texttt{mean()} function,
you can run the following code:

\begin{Shaded}
\begin{Highlighting}[]
\FunctionTok{help}\NormalTok{(mean)}
\end{Highlighting}
\end{Shaded}

You can also use the \texttt{?} operator to get help on a function. For
example, to get help on the \texttt{mean()} function, you can run the
following code:

\begin{Shaded}
\begin{Highlighting}[]

\NormalTok{?mean}
\end{Highlighting}
\end{Shaded}

\hypertarget{what-are-packages}{%
\subsection{What are packages?}\label{what-are-packages}}

R has many built-in functions that you can use to perform tasks.
However, there are also many packages available that contain additional
functions. You can install these packages onto your conputer and then
load them into your R session whenever you want to use them.

To install a package, you can use the \texttt{install.packages()}
function. For example, to install the \texttt{tidyverse} package, you
would run the following code:

\begin{Shaded}
\begin{Highlighting}[]
\FunctionTok{install.packages}\NormalTok{(}\StringTok{"tidyverse"}\NormalTok{)}
\end{Highlighting}
\end{Shaded}

To load a package into your R session, you can use the
\texttt{library()} function. For example, to load the \texttt{tidyverse}
package, you would run the following code:

\begin{Shaded}
\begin{Highlighting}[]
\FunctionTok{library}\NormalTok{(tidyverse)}
\end{Highlighting}
\end{Shaded}

Once you have loaded a package, you can use the functions in that
package in your code. For example, the \texttt{tidyverse} package
contains functions for data manipulation and visualisation.

\bookmarksetup{startatroot}

\hypertarget{working-with-data-in-r-studio}{%
\chapter{Working with data in R
Studio}\label{working-with-data-in-r-studio}}

Working with your data in RStudio is a little bit different from working
with data in a spreadsheet, for example. One crucial difference is that
you need to be explicit about what you want to do with your data. In a
spreadsheet, you can simply click on a cell and start typing. In R, you
need to tell the software what you want to do with your data. This can
be a bit intimidating at first, but it is also one of the most powerful
features of R. It allows you to automate repetitive tasks and perform
complex analyses with just a few lines of code.

\hypertarget{importing-data-into-r}{%
\section{Importing data into R}\label{importing-data-into-r}}

\begin{tcolorbox}[enhanced jigsaw, opacitybacktitle=0.6, rightrule=.15mm, coltitle=black, bottomrule=.15mm, breakable, titlerule=0mm, leftrule=.75mm, toprule=.15mm, arc=.35mm, colback=white, colframe=quarto-callout-tip-color-frame, opacityback=0, bottomtitle=1mm, toptitle=1mm, title=\textcolor{quarto-callout-tip-color}{\faLightbulb}\hspace{0.5em}{At the end of this section, you will be able to:}, left=2mm, colbacktitle=quarto-callout-tip-color!10!white]

\begin{itemize}
\tightlist
\item
  Load data into R from different file types
\item
  Understand the structure of data in R
\end{itemize}

\end{tcolorbox}

There are a few different ways to load data into R. You can load data
from a file on your computer, from a URL, or from a package. You can
load data in different file types, such as CSV, Excel, and SPSS files.

Using RStudio, you can load data by clicking on
\texttt{File\ \textgreater{}\ Import\ Dataset}. This will open a window
where you can select the file that you want to load.

However, you can also load data using code. For example, you can use the
\texttt{read\_csv()} function from the \texttt{readr} package to load a
CSV file into R. You can use the \texttt{readxl} package to load an
Excel file, and the \texttt{haven} package to load an SPSS file.

\begin{Shaded}
\begin{Highlighting}[]

\CommentTok{\# Load the readr package}

\FunctionTok{library}\NormalTok{(readr)}

\CommentTok{\# Load a CSV file into R}

\NormalTok{data }\OtherTok{\textless{}{-}} \FunctionTok{read\_csv}\NormalTok{(}\StringTok{"data.csv"}\NormalTok{)}
\end{Highlighting}
\end{Shaded}

Let's break this code down:

\begin{itemize}
\item
  \texttt{library(readr)} loads the \texttt{readr} package into your R
  session. This package contains the \texttt{read\_csv()} function,
  which you can use to load a CSV file into R.
\item
  \texttt{read\_csv("data.csv")} reads the CSV file called
  \texttt{data.csv} into R. The data will be stored in an object called
  \texttt{data}.
\end{itemize}

When you load data into R, it will be stored as a data frame. A data
frame is a type of object in R that is used to store tabular data. It is
similar to a spreadsheet in Excel, with rows and columns.

\hypertarget{how-are-data-stored-in-r}{%
\section{How are data stored in R?}\label{how-are-data-stored-in-r}}

If you worked through the previous section, you should already have some
idea how to load data into R. But how are data stored in R? In R, data
are stored in objects. An object is a container that holds data. There
are several types of objects in R, but the most common ones are:

\begin{itemize}
\tightlist
\item
  Vectors (e.g., a sequence of numbers)
\item
  Matrices (e.g., a table of rows and colummns, all of the same data
  type)
\item
  Data frames (e.g., a table of data where each column represents a
  variable and each row represents an observation)
\item
  Lists (e.g., a collection of objects)
\end{itemize}

In this section, we will focus on data frames, which are the most common
way to store data in R. A data frame is a table of data where each
column represents a variable and each row represents an observation. You
can think of a data frame as having a structure similar to a
spreadsheet.

\begin{figure}

{\centering \includegraphics{images/dataframe.png}

}

\caption{Data frame with 3 variables/columns}

\end{figure}

\hypertarget{how-do-we-use-data-frames-in-r}{%
\section{How do we use data frames in
R?}\label{how-do-we-use-data-frames-in-r}}

To view the data in a data frame, you can simply type the name of the
data frame in the console and press Enter. For example, if you have a
data frame called \texttt{my\_data}, you can view the data in the data
frame by typing \texttt{my\_data} in the console and pressing Enter.

\begin{Shaded}
\begin{Highlighting}[]
\DocumentationTok{\#\# load the tidyverse package}

\FunctionTok{library}\NormalTok{(tidyverse)}

\CommentTok{\# Create a data frame}
\NormalTok{my\_data }\OtherTok{\textless{}{-}} \FunctionTok{data.frame}\NormalTok{(}
  \AttributeTok{name =} \FunctionTok{c}\NormalTok{(}\StringTok{"Alice"}\NormalTok{, }\StringTok{"Bob"}\NormalTok{, }\StringTok{"Charlie"}\NormalTok{, }\StringTok{"David"}\NormalTok{, }\StringTok{"Eve"}\NormalTok{, }\StringTok{"Frank"}\NormalTok{),}
  \AttributeTok{age =} \FunctionTok{c}\NormalTok{(}\DecValTok{25}\NormalTok{, }\DecValTok{30}\NormalTok{, }\DecValTok{35}\NormalTok{, }\DecValTok{40}\NormalTok{, }\DecValTok{45}\NormalTok{, }\DecValTok{50}\NormalTok{),}
  \AttributeTok{height =} \FunctionTok{c}\NormalTok{(}\DecValTok{160}\NormalTok{, }\DecValTok{175}\NormalTok{, }\DecValTok{180}\NormalTok{, }\DecValTok{165}\NormalTok{, }\DecValTok{170}\NormalTok{, }\DecValTok{190}\NormalTok{),}
  \AttributeTok{car =} \FunctionTok{c}\NormalTok{(}\StringTok{"Electric"}\NormalTok{, }\StringTok{"Petrol"}\NormalTok{, }\StringTok{"Electric"}\NormalTok{, }\StringTok{"Petrol"}\NormalTok{, }\StringTok{"Petrol"}\NormalTok{, }\StringTok{"Electric"}\NormalTok{)}
\NormalTok{)}

\CommentTok{\# View or refer to the data in the data frame}

\NormalTok{my\_data}
\end{Highlighting}
\end{Shaded}

\begin{verbatim}
     name age height      car
1   Alice  25    160 Electric
2     Bob  30    175   Petrol
3 Charlie  35    180 Electric
4   David  40    165   Petrol
5     Eve  45    170   Petrol
6   Frank  50    190 Electric
\end{verbatim}

In the code above, we created a data frame called \texttt{my\_data} with
four variables: \texttt{name}, \texttt{age}, \texttt{height}, and
\texttt{car}. We then used the \texttt{my\_data} object to view the data
in the data frame.

\hypertarget{view-or-refer-to-a-specific-variable-in-a-data-frame}{%
\section{View or refer to a specific variable in a data
frame}\label{view-or-refer-to-a-specific-variable-in-a-data-frame}}

To view or refer to a specific variable in a data frame, you can use the
\texttt{\$} operator. For example, if you want to view the \texttt{age}
variable in the \texttt{my\_data} data frame, you can type
\texttt{my\_data\$age} in the console and press Enter.

\begin{Shaded}
\begin{Highlighting}[]
\CommentTok{\# View or refer to a specific variable in a data frame}

\NormalTok{my\_data}\SpecialCharTok{$}\NormalTok{age}
\end{Highlighting}
\end{Shaded}

\begin{verbatim}
[1] 25 30 35 40 45 50
\end{verbatim}

\hypertarget{data-types-in-r}{%
\section{Data types in R}\label{data-types-in-r}}

In R, each variable in a data frame has a data type. The most common
data types in R are:

\begin{itemize}
\tightlist
\item
  Numeric: for continuous variables (e.g., age, height)
\item
  Factor: for categorical variables
\item
  Logical: for binary variables (TRUE or FALSE)
\end{itemize}

You can use the \texttt{str()} function to view the structure of a data
frame, including the data types of each variable.

\begin{Shaded}
\begin{Highlighting}[]
\CommentTok{\# View the structure of a data frame}

\FunctionTok{str}\NormalTok{(my\_data)}
\end{Highlighting}
\end{Shaded}

\begin{verbatim}
'data.frame':   6 obs. of  4 variables:
 $ name  : chr  "Alice" "Bob" "Charlie" "David" ...
 $ age   : num  25 30 35 40 45 50
 $ height: num  160 175 180 165 170 190
 $ car   : chr  "Electric" "Petrol" "Electric" "Petrol" ...
\end{verbatim}

In the code above, we used the \texttt{str()} function to view the
structure of the \texttt{my\_data} data frame. The output shows the data
types of each variable in the data frame.

\hypertarget{convert-data-types-in-r}{%
\section{Convert data types in R}\label{convert-data-types-in-r}}

You can convert the data type of a variable in R using the \texttt{as.}
functions. For example, you can convert a character variable to a factor
variable using the \texttt{as.factor()} function.

\begin{Shaded}
\begin{Highlighting}[]
\CommentTok{\# Convert a character variable to a factor variable}

\NormalTok{my\_data}\SpecialCharTok{$}\NormalTok{name }\OtherTok{\textless{}{-}} \FunctionTok{as.factor}\NormalTok{(my\_data}\SpecialCharTok{$}\NormalTok{name)}

\NormalTok{my\_data}\SpecialCharTok{$}\NormalTok{car }\OtherTok{\textless{}{-}} \FunctionTok{as.factor}\NormalTok{(my\_data}\SpecialCharTok{$}\NormalTok{car)}
\end{Highlighting}
\end{Shaded}

In the code above, we converted the \texttt{name} variable in the
\texttt{my\_data} data frame from a character variable to a factor
variable using the \texttt{as.factor()} function.

\hypertarget{subsetting-data-in-r}{%
\section{Subsetting data in R}\label{subsetting-data-in-r}}

\begin{tcolorbox}[enhanced jigsaw, opacitybacktitle=0.6, rightrule=.15mm, coltitle=black, bottomrule=.15mm, breakable, titlerule=0mm, leftrule=.75mm, toprule=.15mm, arc=.35mm, colback=white, colframe=quarto-callout-tip-color-frame, opacityback=0, bottomtitle=1mm, toptitle=1mm, title=\textcolor{quarto-callout-tip-color}{\faLightbulb}\hspace{0.5em}{At the end of this section, you will be able to:}, left=2mm, colbacktitle=quarto-callout-tip-color!10!white]

\begin{itemize}
\tightlist
\item
  Filter data in R
\item
  Create subsets of data in R
\end{itemize}

\end{tcolorbox}

Subsetting data in R means selecting a subset of the data based on
certain criteria. For example, you might want to select only the rows
where a certain variable is greater than a certain value, or only the
columns that contain certain variables.

If we use the \texttt{my\_data} data frame from the previous section, we
can subset the data to select only the rows where the \texttt{age}
variable is greater than 30.

\begin{Shaded}
\begin{Highlighting}[]
\CommentTok{\# Filter the data frame to select only the rows where the age variable is greater than 30}

\CommentTok{\# this method uses the dplyr package, which is a part of the tidyverse. Be sure to load the tidyverse package if you haven\textquotesingle{}t already.}

\NormalTok{my\_data }\SpecialCharTok{\%\textgreater{}\%} \FunctionTok{filter}\NormalTok{(age }\SpecialCharTok{\textgreater{}} \DecValTok{30}\NormalTok{)}
\end{Highlighting}
\end{Shaded}

\begin{verbatim}
     name age height      car
1 Charlie  35    180 Electric
2   David  40    165   Petrol
3     Eve  45    170   Petrol
4   Frank  50    190 Electric
\end{verbatim}

Let's break this code down:

\begin{itemize}
\tightlist
\item
  \texttt{my\_data} is the data frame that we want to subset.
\item
  \texttt{\%\textgreater{}\%} is the pipe operator, which is used to
  pass the data frame to the next function. This allows us to link
  multiple steps together in a single line of code.
\item
  \texttt{filter(age\ \textgreater{}\ 30)} is the function that filters
  the data frame to select only the rows where the \texttt{age} variable
  is greater than 30.
\end{itemize}

The output of this code will be a new data frame that contains only the
rows where the \texttt{age} variable is greater than 30. However, this
new data frame will not be saved anywhere, so if you want to save it,
you need to assign it to a new object. Tp do this, you can use the
assignment operator \texttt{\textless{}-}.

\begin{Shaded}
\begin{Highlighting}[]
\CommentTok{\# Filter the data frame to select only the rows where the age variable is greater than 30 and save the result to a new data frame called new\_data}

\NormalTok{new\_data }\OtherTok{\textless{}{-}}\NormalTok{ my\_data }\SpecialCharTok{\%\textgreater{}\%} \FunctionTok{filter}\NormalTok{(age }\SpecialCharTok{\textgreater{}} \DecValTok{30}\NormalTok{)}
\end{Highlighting}
\end{Shaded}

In this code, on the left side of the assignment operator
\texttt{\textless{}-}, we have \texttt{new\_data}, which is the name of
the new data frame that will contain only the filtered subset of the
data (i.e., the values where the \texttt{age} variable is greater than
30). The difference between this code and the previous code is that we
are now saving the result to a new data frame called \texttt{new\_data},
instead of just printing it to the console.

We can also combine multiple conditions when subsetting data. For
example, we can select only the rows where the \texttt{age} variable is
greater than 25 and the \texttt{height} variable is greater than 175.

\begin{Shaded}
\begin{Highlighting}[]
\CommentTok{\# Filter the data frame to select only the rows where the age variable is greater than 25 and the height variable is greater than 175}

\NormalTok{my\_data }\SpecialCharTok{\%\textgreater{}\%} \FunctionTok{filter}\NormalTok{(age }\SpecialCharTok{\textgreater{}} \DecValTok{25} \SpecialCharTok{\&}\NormalTok{ height }\SpecialCharTok{\textgreater{}} \DecValTok{175}\NormalTok{)}
\end{Highlighting}
\end{Shaded}

\begin{verbatim}
     name age height      car
1 Charlie  35    180 Electric
2   Frank  50    190 Electric
\end{verbatim}

There are many other ways to subset data in R, depending on the criteria
you want to use. For example, you can use the \texttt{select()} function
to select specific columns, the \texttt{arrange()} function to sort the
data, and the \texttt{mutate()} function to create new variables. We
will cover some of these functions in later sections.

\hypertarget{grouping-and-summarising-data-in-r}{%
\section{Grouping and summarising data in
R}\label{grouping-and-summarising-data-in-r}}

\begin{tcolorbox}[enhanced jigsaw, opacitybacktitle=0.6, rightrule=.15mm, coltitle=black, bottomrule=.15mm, breakable, titlerule=0mm, leftrule=.75mm, toprule=.15mm, arc=.35mm, colback=white, colframe=quarto-callout-tip-color-frame, opacityback=0, bottomtitle=1mm, toptitle=1mm, title=\textcolor{quarto-callout-tip-color}{\faLightbulb}\hspace{0.5em}{At the end of this section, you will be able to:}, left=2mm, colbacktitle=quarto-callout-tip-color!10!white]

\begin{itemize}
\tightlist
\item
  Group data in R
\item
  Summarise data in R
\end{itemize}

\end{tcolorbox}

Grouping and summarising data in R means grouping the data by one or
more variables and then calculating summary statistics for each group.
For example, you might want to calculate the mean age for each group of
people based on their height.

If we use the \texttt{my\_data} data frame from the previous section, we
can group the data by the \texttt{car} variable and then calculate the
mean age for each group.

\begin{Shaded}
\begin{Highlighting}[]
\CommentTok{\# Group the data frame by the car variable and calculate the mean age for each group}

\NormalTok{my\_data }\SpecialCharTok{\%\textgreater{}\%} \FunctionTok{group\_by}\NormalTok{(car) }\SpecialCharTok{\%\textgreater{}\%} 
  \FunctionTok{summarise}\NormalTok{(}\AttributeTok{mean\_age =} \FunctionTok{mean}\NormalTok{(age)) }\SpecialCharTok{\%\textgreater{}\%}
  \FunctionTok{ungroup}\NormalTok{()}
\end{Highlighting}
\end{Shaded}

\begin{verbatim}
# A tibble: 2 x 2
  car      mean_age
  <fct>       <dbl>
1 Electric     36.7
2 Petrol       38.3
\end{verbatim}

Let's break this code down:

\begin{itemize}
\tightlist
\item
  \texttt{my\_data} is the data frame that we want to group and
  summarise.
\item
  \texttt{\%\textgreater{}\%} is the pipe operator, which is used to
  pass the data frame to the next function. This allows us to link
  multiple steps together in a single line of code.
\item
  \texttt{group\_by(car)} is the function that groups the data frame by
  the \texttt{car} variable.
\item
  \texttt{summarise(mean\_age\ =\ mean(age))} is the function that
  calculates the mean age for each group of cars. The \texttt{mean\_age}
  variable is the name of the new variable that will contain the mean
  age for each group.
\item
  \texttt{ungroup()} is the function that removes the grouping from the
  data frame. This is optional, but it is good practice to ungroup the
  data frame after you have finished summarising it.
\end{itemize}

The output of this code will be a new data frame that contains the mean
age for each group of cars. The \texttt{car} variable is the grouping
variable, and the \texttt{mean\_age} variable is the summary statistic
that we calculated for each group.

You can also calculate other summary statistics, such as the median,
standard deviation, minimum, and maximum, using the \texttt{summarise()}
function. You can also calculate multiple summary statistics at the same
time by specifying multiple variables inside the \texttt{summarise()}
function. For example, you can calculate the mean and standard deviation
of the age variable for each group of cars.

\begin{Shaded}
\begin{Highlighting}[]
\CommentTok{\# Group the data frame by the car variable and calculate the mean and standard deviation of the age variable for each group}

\NormalTok{my\_data }\SpecialCharTok{\%\textgreater{}\%} \FunctionTok{group\_by}\NormalTok{(car) }\SpecialCharTok{\%\textgreater{}\%} 
  \FunctionTok{summarise}\NormalTok{(}\AttributeTok{mean\_age =} \FunctionTok{mean}\NormalTok{(age), }\AttributeTok{sd\_age =} \FunctionTok{sd}\NormalTok{(age)) }\SpecialCharTok{\%\textgreater{}\%}
  \FunctionTok{ungroup}\NormalTok{()}
\end{Highlighting}
\end{Shaded}

\begin{verbatim}
# A tibble: 2 x 3
  car      mean_age sd_age
  <fct>       <dbl>  <dbl>
1 Electric     36.7  12.6 
2 Petrol       38.3   7.64
\end{verbatim}

In this code, we calculated the mean and standard deviation of the
\texttt{age} variable for each group of cars. The \texttt{mean\_age} and
\texttt{sd\_age} variables are the names of the new variables that will
contain the mean and standard deviation for each group.

\bookmarksetup{startatroot}

\hypertarget{references}{%
\chapter*{References}\label{references}}
\addcontentsline{toc}{chapter}{References}

\markboth{References}{References}

\hypertarget{refs}{}
\begin{CSLReferences}{0}{0}
\end{CSLReferences}



\end{document}
